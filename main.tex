\documentclass{article}
\usepackage[utf8]{inputenc}

%programming code package
\usepackage{listings}
\usepackage{color}

\definecolor{dkgreen}{rgb}{0,0.6,0}
\definecolor{gray}{rgb}{0.5,0.5,0.5}
\definecolor{mauve}{rgb}{0.58,0,0.82}

\lstset{frame=tb,
  language=Java,
  aboveskip=3mm,
  belowskip=3mm,
  showstringspaces=false,
  columns=flexible,
  basicstyle={\small\ttfamily},
  numbers=none,
  numberstyle=\tiny\color{gray},
  keywordstyle=\color{blue},
  commentstyle=\color{dkgreen},
  stringstyle=\color{mauve},
  breaklines=true,
  breakatwhitespace=true,
  tabsize=3
}


\title{Java COMP401 Class Summary}
\author{Carlos D. Rubido Rafull }
\date{September 2019}

\begin{document}


\maketitle

\section{Primitive Value Types Vs Reference Types}
For this first part all the sample code will be inside the ex1 folder.
\begin{itemize}
    \item Primitive Value Types : Value types are defined entirely by their value.
    
       \textbf{Integers}: byte, short, int, long. Their difference is the size (1,2,4, or 8 bytes). They do not have "unsigned" version.

    
    \begin{itemize}
    \item example
    \begin{lstlisting}
    public static void main(String[] args) {
        byte validByte = 100; // Valid byte
        byte invalidByte = 129; // Invalid byte
        
        short validShort = -32000; // Valid short 
        short invalidShort = -32769; // Invalid short
        
        int validInt = 2147483647; // Valid int
        int invalidInt = 2147483648; // Invalid int
        
        long validLong = -9223372036854775808; // Valid long
        long invalidLong = 9223372036854775808; // Invalid long
    }
    \end{lstlisting}
    \end{itemize}
    
    \textbf{Real numbers}: float, double. The float type is single precision while double is double precision. We will never use float to deal with currency.
    \begin{itemize}
    \item example
    \begin{lstlisting}
    public static void main(String[] args) {
        float validFloat = 3.141593;
        float invalidFloat = 3.141592654;
        
        double validDouble = 3.14159265359;
    }
    \end{lstlisting}
    \end{itemize}
    
    \textbf{Characters}: char. Characters in java are 16 bit unicode value. They are declared using single quotes, or a unicode scape sequence using hex digits. 
    
    \begin{itemize}
    \item example
    \begin{lstlisting}
    public static void main(String[] args) {
        char validChar = 'c';
        char otherValidChar = '\u0063';
        
        char invalidChar = 'abc';
    }
    \end{lstlisting}
    \end{itemize}
    
    \textbf{Booleans}: booleans. They can take true or false values. Their size is not precisely defined.
    \begin{itemize}
    \item example
    \begin{lstlisting}
    public static void main(String[] args) {
        boolean validBool = true;
        boolean otherValidBool = false;
    }
    \end{lstlisting}
    \end{itemize}
    
    \textbf{Check out the \texttt{ValueTypes.java} class to play around with examples of all value types in java and their declaration.}
    
    \item Reference Type : Any Object. String, Arrays, Classes themselves.
    \begin{itemize}
    \item Reference type examples
    \begin{lstlisting}
    public class ReferenceTypes {

    public static void main(String[] args) {
        /*
        The following examples are reference types.
        */

        //Name of a student
        String validString = "Carlos";
        //Array of integers (size 10) to store the ages of 10 students
        int[] age = new int[10]; 
        //Array of doubles (size 20) to store the salary of 20 professors
        double[] salary = new double[20];
        // Any object or any class for example a Person object
        Person newPerson = new Person("John", "Doe", 21);
        // Array of objects for example array of Persons to store 100 persons
        Person[] persons = new Person[100];
    }
}
    \end{lstlisting}
    \end{itemize}
    
    \begin{itemize}
    \item Person class to illustrate reference type examples
    \begin{lstlisting}
    public class Person {

    //Fields of the Person class
    //Every person has a first name, last name, and age
    private String personFirstName;
    private String personLastName;
    private int personAge;

    public Person(String personFirstName, String personLastName, int personAge) {
        this.personFirstName = personFirstName;
        this.personLastName = personLastName;
        this.personAge = personAge;
    }

    public String getPersonName() {
        return this.personFirstName + " " + this.personLastName;
    }

    public int getPersonAge() {
        return this.personAge;
    }

    public void setAge(int age) {
        this.personAge = age;
    }

}
    \end{lstlisting}
     \end{itemize}
     \textbf{Check out the \texttt{ReferenceTypes.java} and \texttt{Person.java} class to play around with examples of all reference types in java and their declaration.}
\end{itemize}


\section{Expressions in Java}
\begin{itemize}

\item A sequence of symbols that can be evaluated to produce an value. Expressions can be used wherever a value is expected.
\newpage
\begin{itemize}
    \item Example :
    \begin{lstlisting}
    import java.lang.Math;

public class Expressions {

    public static void main(String[] args) {
        
        // Literal values
        System.out.println(123);
        System.out.println('c');
        System.out.println("A String");
        System.out.println(true);

        // Named values
        String name = "John";
        System.out.println(name);

        // Value retrived from an array
        int[] myIntegerArray = {10,11,12,13};
        System.out.println(myIntegerArray[0]);

        // Class/object fields
        System.out.println(Math.PI);

        // Value as a result of a method call
        System.out.println(circleArea(2));

        // Note: void methods do not return any value hence a call of a void method is not consider an expression
        johnsName();

        // Compound expressions of operators
        System.out.println(4 + 3*2);
    }

    public static double circleArea(double r) {
        return Math.PI * Math.pow(r, 2);
    }

    public static void johnsName() {
        System.out.println("John Doe");
    }
}
    \end{lstlisting}
\end{itemize}

\end{itemize}

\end{document}
